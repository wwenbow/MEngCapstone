%        File: Intro.tex
%     Created: Sun Mar 02 03:00 PM 2014 P
% Last Change: Sun Mar 02 03:00 PM 2014 P
%
\documentclass{article}
\usepackage{fullpage}
\usepackage{hyperref}
\usepackage{url}

\begin{document}

\section{Introduction}
\subsection{Current Technology}
\paragraph{}
Currently many manufacturing robots do not utilize computer vision to
accomplish their tasks. Many manufacturing robots simply run a simple preset
routine without much intellect driving them. As computational power becomes
cheaper and computer vision algorithms become more mature, these new
technologies can offer much needed improvements to manufacturing robotics. High
tech areas, such as computer vision, can be very profitable targets for our
sponsor, Bay Area IP, LLC. Our sponsor seeks to generate intellectual property
to license out to companies in the robotics field, so it will be profitable to
focus on the high tech areas. 

\paragraph{} There exist some manufacturing robots on the market that are
utilizing computer vision to accomplish complex production line tasks. One such
robot is produced by Italy based SIR SpA. The robot utilizes stereoscopic 3D
vision systems to find and pick up parts from an unstructured bin. However,
this system utilizes a large camera and complex algorithms to find and detect
individual parts in a disorganized bin full of such parts\cite{SIRfuture}. Our
Capstone technology seeks to provide a simpler computer vision solution for
less complex object detection problems.

\paragraph{} Computer vision technology has many tools for object detection and
recognition.  There are many machine learning paradigms that can be applied to
the task of object detection. Neural Networks and Support Vector Machines(SVM)
are two technologies that are commonly used in this area. Neural Networks are
very complex to tweak and configure, but can achieve great accuracy while being
light weight at run time. SVM is a well understood mathematically and
configuration of such systems is also well understood. They are light weight if
used with simple linear kernels, but more complex kernels would be difficult to
implement on a robot\cite{SVMStackOverflow}. However, all these technologies
can be difficult to implement on an embedded system such as a robot, so we will
attempt to do develop new techniques that will be lighter on computational
power.

\paragraph{} The other component of our robot is the robotic hand. There are
many different companies and research groups constructing robotic hands for
many different applications. There are hands designed to be prosthetics, as
well as hands designed to be integrated into robotic systems. One of the most
complex is the UB Hand IV, this hand is nearly the same as a human hand in
terms of capabilities. However, the hand is very heavy and very power hungry,
so it is not very suitable for most applications\cite{Melchiorri2013}. We want
to leverage some modern technology to make a robotic hand that is smaller and
light weight.

\subsection{Capstone Technology} \paragraph{} We will be attempting to produce
a cheap and simple computer vision technology for object detection. Accomplish
this, we will be utilizing lasers to make the image processing challenges
simpler. Processing live video using common machine learning based computer
vision algorithms may prove to be challenging for the smaller computers we will
be utilizing for the robot. To help make the task simpler, we will utilize a
laser on the head to scan the space. By using the laser to scan the image, and
utilizing the live video of the laser motion, we can easily detect edges
without using complex algorithms. Using these edges we can find objects. We
will also have another laser on the arm. Once we have an object detected, we
can use the laser on the head to point at the object, and try to line up the
arm using it's own laser. By using the two lasers we can try to guide the arm
towards the object.

\paragraph{} To make the robotic hand, we will be trying to use small
components as much as possible to reduce the size and weight. By using smaller
motors, we can make the hand small and still have many degrees of freedom. We
cannot make the hand as complex as a human hand, so we will need to sacrifice
some capabilities to make the hand small and low power. A major technology that
we plan to implement is Shape Memory Alloy(SMA). SMA is a new material that can
contract when heated this allows us to use it as an actuator. We can utilize
wires of SMA to control joints and other areas of the hand where space is very
limited. Combining these various technologies, we can make a hand that is very
human-like.

\bibliographystyle{plain}
\bibliography{/home/wenbo/Dropbox/UCB/295/CapstoneCitations}

\end{document}


