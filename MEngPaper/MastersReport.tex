%        File: Intro.tex
%     Created: Sun Mar 02 03:00 PM 2014 P
% Last Change: Sun Mar 02 03:00 PM 2014 P
%
\documentclass[letterpaper]{article}
\usepackage{biblatex}
\usepackage{fullpage}
\usepackage{hyperref}
\usepackage{amsmath}
%\usepackage[doublespacing]{setspace}
\addbibresource{CapstoneCitations.bib}

\pagestyle{plain}

\title{Humanoid Robotic Subsystems}
\date{April 14, 2014}
\author{Wenbo Wang \\University of California, Berkeley}

\begin{document}

\maketitle

\begin{abstract}
There is a growing need for manufacturing robots in many different sectors of
industry. As demand increases and technology improves there will be a great
demand for robots that can better integrate into the workplace. We propose two
new areas of Intellectual Property opportunities for Bay Area IP, LLC: Computer
Vision and Robotic Hands. There will be a great demand for robots that have high
precision and can better interact with human co-workers, as current
manufacturing robots are large machinery that are often not safe for human
workers. Computer vision allows the robot to be more aware of their surrounding,
and perform more tasks precisely. Humanoid hands also allow for more precision
and better integration with tools workspaces designed for humans. Are
preliminary results show great difficulty in building suitable robotic hands
with high dexterity without many the device large and cumbersome, but simple
computer vision can be built cheaply and effectively.  
\end{abstract}

\section{Introduction}
\paragraph{}Robots are being widely adopted by in many industries, and there is
a growing need for humanoid robotics. Humanoid robotic components can have many
applications such as manufacturing and prosthetics. Currently humanoid robots
and robot subsystems are not in wide use, so it is a good target for our
sponsor, Bay Area IP, which is looking do develop intellectual property that
they could license to others.

\paragraph{}There are many technical challenges that come with trying to
reproduce the human form. The human body is very complex and there are many
different components that need to interact and function correctly. The human
hand has 27 degrees of freedom, and it is very difficult to reproduce that;
however, modern manufacturing techniques such as 3D printing can get us closer
to reproducing that complexity. 

\paragraph{}The subsystems we designed were the arms, hands, legs, feet, and
vision systems.  We designed the mechanical structure of these components and
the software for controlling the different parts. 

\section{Literature Review}
\subsection{Current Technology}
\paragraph{}Currently many manufacturing robots do not utilize computer vision to
accomplish their tasks. Many manufacturing robots simply run a simple preset
routine without much intellect driving them. As computational power becomes
cheaper and computer vision algorithms become more mature, these new
technologies can offer much needed improvements to manufacturing robotics. High
tech areas, such as computer vision, can be very profitable targets for our
sponsor, Bay Area IP, LLC. Our sponsor seeks to generate intellectual property
to license out to companies in the robotics field, so it will be profitable to
focus on the high tech areas. 

\paragraph{} There exist some manufacturing robots on the market that are
utilizing computer vision to accomplish complex production line tasks. One such
robot is produced by Italy based SIR SpA. The robot utilizes stereoscopic 3D
vision systems to find and pick up parts from an unstructured bin. However,
this system utilizes a large camera and complex algorithms to find and detect
individual parts in a disorganized bin full of such parts\cite{SIRfuture}. Our
Capstone technology seeks to provide a simpler computer vision solution for
less complex object detection problems.

\paragraph{} Computer vision technology has many tools for object detection and
recognition.  There are many machine learning paradigms that can be applied to
the task of object detection. Neural Networks and Support Vector Machines(SVM)
are two technologies that are commonly used in this area. Neural Networks are
very complex to tweak and configure, but can achieve great accuracy while being
light weight at run time. SVM is a well understood mathematically and
configuration of such systems is also well understood. They are light weight if
used with simple linear kernels, but more complex kernels would be difficult to
implement on a robot\cite{SVMStackOverflow}. However, all these technologies
can be difficult to implement on an embedded system such as a robot, so we will
attempt to do develop new techniques that will be lighter on computational
power.

\paragraph{} The other component of our robot is the robotic hand. There are
many different companies and research groups constructing robotic hands for
many different applications. There are hands designed to be prosthetics, as
well as hands designed to be integrated into robotic systems. One of the most
complex is the UB Hand IV, this hand is nearly the same as a human hand in
terms of capabilities. However, the hand is very heavy and very power hungry,
so it is not very suitable for most applications\cite{Melchiorri2013}. We want
to leverage some modern technology to make a robotic hand that is smaller and
light weight.

\subsection{Capstone Technology} \paragraph{} We will be attempting to produce
a cheap and simple computer vision technology for object detection. Accomplish
this, we will be utilizing lasers to make the image processing challenges
simpler. Processing live video using common machine learning based computer
vision algorithms may prove to be challenging for the smaller computers we will
be utilizing for the robot. To help make the task simpler, we will utilize a
laser on the head to scan the space. By using the laser to scan the image, and
utilizing the live video of the laser motion, we can easily detect edges
without using complex algorithms. Using these edges we can find objects. We
will also have another laser on the arm. Once we have an object detected, we
can use the laser on the head to point at the object, and try to line up the
arm using it's own laser. By using the two lasers we can try to guide the arm
towards the object.

\paragraph{} To make the robotic hand, we will be trying to use small
components as much as possible to reduce the size and weight. By using smaller
motors, we can make the hand small and still have many degrees of freedom. We
cannot make the hand as complex as a human hand, so we will need to sacrifice
some capabilities to make the hand small and low power. A major technology that
we plan to implement is Shape Memory Alloy(SMA). SMA is a new material that can
contract when heated this allows us to use it as an actuator. We can utilize
wires of SMA to control joints and other areas of the hand where space is very
limited. Combining these various technologies, we can make a hand that is very
human-like.


\section{Materials and Methods}
\subsection{Materials}
\paragraph{PIC32MX795F512L:}A 32 bit micro-controller for performing the control
algorithms of the robot. The PIC32MX795F512L comes from SparkFun electronics as
part of the UBW32 board, which comes preloaded with a bootloader for programming
the board. The PIC32MX795F512L is programmed in C/C++. To make the device easier
to use, we loaded an avrdude bootloader onto the device to use the MPIDE
software. The MPIDE framework allows us to use Arduino libraries on the
PIC32MX795F512L which helps to accelerate development of basic I/O software for
the board\cite{pic32data}.

\paragraph{PICKIT3:}A hardware programmer for the PIC32MX795F512L
microprocessor. The pickit3 and program and debug the PIC32 micro-controller.
Used to download code onto the UBW32. Useful for installing new bootloaders onto
the UBW32 and for restoring broken firmware\cite{pickitdata}.

\paragraph{MPU6050:}A gyro and accelerometer for obtaining information about the
kinematics of the arm and the leg. Uses I2C connection to the PIC32MX795F512L
for communication\cite{mpu6050data}.

\paragraph{HSR-5498SG:}Servos from Hitec. Utilized to control arm and leg
joints.  The servos require 6-7 Volts. Each servo requires approximately 200mA.
The actuation of the arm is accomplished by mini-motors\cite{sscdata}.

\paragraph{SSC-32:}Servo controller from Lynx motion. The on-board controller
is an Atmega168-20PU. The SSC-32 is controlled using serial signals from another
microprocessor or the PC. The control signal is a simple byte stream. The
controller can specify pin number, servo position, rotation speed, and rotation
time. It can power 32 servos simultaneously\cite{servodata}.

\subsection{Methods}
\subsubsection{Preliminary Setup}
\paragraph{}To program the PIC32MX795F512L, we used the Multi-Platform
Integrated Development Environment(MPIDE) environment from chipKIT. To use MPIDE
with the program, we had to program the PIC32 with the avrdude bootloader found
here: \url{https://github.com/chipKIT32/PIC32-avrdude-bootloader}. Using the
pickit3, we loaded the new bootloader onto the UBW32 and was able to use MPIDE
to program the board.

\paragraph{}While the other team members were constructing the mechanical
components of the legs and arms, I proceeded to set up the basic components of
the controller. We set up the MPU6050 using the PIC32 I2C libraries in MPIDE and
proceeded to do basic calibration and testing. Then we set up the servos using
the SSC-32 controller and the PIC32. Using the PIC32's Universal Asynchronous
Receiver/Transmitter(UART) we set up a serial communication channel between the
PIC32 and the SSC-32.  

\subsubsection{Servo Setup} 
\paragraph{} To control the servos we would send a string through the PIC32 UART
channel to the SSC-32. The string encoded the pin number, desired position and
desired rotation time for one of the 32 servo channels on the
SSC-32\cite{sscdata}. There were six servos on each leg of the robot and four on
each arm. On the legs, there were two servos per ankle, one on the knee, and
three on the hip. In total that gives us six degrees of freedom.  The arms were
assembled similarly, with one servo on the elbow and three in the shoulder. The
servos were powered through the SSC-32 board. Each servo requires around 200mA
of current and 6-7V of voltage\cite{servodata}. We used a 6V power supply to
power the servos channels on the SSC-32, and a 9V battery to power the logic
circuit of the servos. The PIC32 was powered through a USB connection to the PC.

\subsubsection{Walking Algorithm Implementation} 
\paragraph{} The walking algorithms were designed by Zhu Ziqi, a fellow team
member. He designed and simulated the algorithms in MATALB and SIMULINK and I
implemented the algorithms in C++ on the PIC32MX795F512L. The algorithms are
based on Central Pattern Generators, a part of biological neural networks which
generate rhythmic control signal for human motion\cite{cpggeneral}. The servos
are controlled using sine waves of varying shapes to generate patterns for
walking. Ziqi performed simulations to to select the best parameters for the
sine waves. For our preliminary walking algorithms we only considered the knee
and hip joints, keeping the other joints rigid. The SSC-32 position control
signal requires a integer ranged from 500 to 2500, corresponding to
$180^{\circ}$ of motion\cite{sscdata}. Due to the mechanical construction of the
arm and leg joints, which mimicked human structure, most joint servos were
limited to a range between 1250 and 2500. The output from the sine waves would
be converted to a range in the SSC-32's output and then sent to the servos. 
\begin{align}
    sine_{knee\_joint}=sin(eqn\_pending) \label{sineknee}\\
    sine_{hip\_joint}=sin(eqn\_pending) \label{sinehip}
\end{align}
Equations \ref{sineknee} and \ref{sinehip} show the equations for calculating
the position of each servo as a function of time. The positions of the legs are
calculated by the PIC32 and sent to the SSC-32.

\subsubsection{Leg Balance Algorithm}
%TODO: Get Yanchen's algorithms
\paragraph{} Waiting for team member to finish simulations of control
algorithms. Likely will use the MPU6050 to get the information about balance.

\subsubsection{Arm Control Algorithm}
\paragraph{} The arms are controlled using a Fuzzy Inference System. A fuzzy
inference system comprises of a fuzzifier, inference system, and defuzzifier.
The fuzzyfier takes inputs and out converts them to fuzzy values of a set of
fuzzy variables for the inference system. In this case the input is the desired
position of the arm in 3D space. The inference system is a set of if else
statements that determine a desired output from the fuzzy values. The
defuzzifier then converts this output to a real output to our
servos\cite{fuzzy2011}. To design this system we will take two approaches, one
through simulation and one through machine learning. I will work on the Machine
Learning approach. MATLAB has neural network algorithms that will generate fuzzy
inference systems. To do this we will generate a dataset based on the forward
kinematics of the robot arm, by giving it the angles of the arm joints and
calculating the arm position as a result of the joint angles. Inputing this
dataset into a machine learning framework in MATLAB, we can generate a simple
fuzzy inference system to control the robotic arm\cite{fuzzymatlab}. To further
refine the control algorithm, we will need to hand tune and simulate the fuzzy
inference system, as well as utilize gyro, accelerometer, and IR sensors to help
guide the arm and avoid obstacles.

\subsubsection{Computer Vision}
\paragraph{} To be completed\dots uses serial communication from APU to PIC32 to
transmit desired arm coordinates.

\subsubsection{Control of Hand}
\paragraph{} Not sure if implementation of hand control is needed for project.

\printbibliography

\end{document}
