%        File: Intro.tex
%     Created: Sun Mar 02 03:00 PM 2014 P
% Last Change: Sun Mar 02 03:00 PM 2014 P
%
\documentclass{article}
\usepackage{fullpage}

\begin{document}

\section{Materials and Methods}
\subsection{Materials}
\paragraph{}
\textbf{PIC32MX795F512L:} A 32 bit micro-controller for performing the control
algorithms of the robot. The PIC32MX795F512L comes from SparkFun electronics as
part of the UBW32 board, which comes preloaded with a bootloader for programming
the board. The PIC32MX795F512L is programmed in C/C++. To make the device easier
to use, we loaded an avrdude bootloader onto the device to use the MPIDE
software. The MPIDE framework allows us to use Arduino libraries on the
PIC32MX795F512L which helps to accelerate development of basic I/O software for
the board\cite{pic32datasheet}. \\

\textbf{MPU6050:} A gyro and accelerometer for obtaining information about the
kinematics of the arm and the leg. Uses I2C connection to the PIC32MX795F512L
for communication.

\paragraph{} Computer vision technology has many tools for object detection and
recognition.  There are many machine learning paradigms that can be applied to
the task of object detection. Neural Networks and Support Vector Machines(SVM)
are two technologies that are commonly used in this area. Neural Networks are
very complex to tweak and configure, but can achieve great accuracy while being
light weight at run time. SVM is a well understood mathematically and
configuration of such systems is also well understood. They are light weight if
used with simple linear kernels, but more complex kernels would be difficult to
implement on a robot\cite{SVMStackOverflow}. However, all these technologies
can be difficult to implement on an embedded system such as a robot, so we will
attempt to do develop new techniques that will be lighter on computational
power.

\paragraph{} The other component of our robot is the robotic hand. There are
many different companies and research groups constructing robotic hands for
many different applications. There are hands designed to be prosthetics, as
well as hands designed to be integrated into robotic systems. One of the most
complex is the UB Hand IV, this hand is nearly the same as a human hand in
terms of capabilities. However, the hand is very heavy and very power hungry,
so it is not very suitable for most applications\cite{Melchiorri2013}. We want
to leverage some modern technology to make a robotic hand that is smaller and
light weight.

\subsection{Capstone Technology} \paragraph{} We will be attempting to produce
a cheap and simple computer vision technology for object detection. Accomplish
this, we will be utilizing lasers to make the image processing challenges
simpler. Processing live video using common machine learning based computer
vision algorithms may prove to be challenging for the smaller computers we will
be utilizing for the robot. To help make the task simpler, we will utilize a
laser on the head to scan the space. By using the laser to scan the image, and
utilizing the live video of the laser motion, we can easily detect edges
without using complex algorithms. Using these edges we can find objects. We
will also have another laser on the arm. Once we have an object detected, we
can use the laser on the head to point at the object, and try to line up the
arm using it's own laser. By using the two lasers we can try to guide the arm
towards the object.

\paragraph{} To make the robotic hand, we will be trying to use small
components as much as possible to reduce the size and weight. By using smaller
motors, we can make the hand small and still have many degrees of freedom. We
cannot make the hand as complex as a human hand, so we will need to sacrifice
some capabilities to make the hand small and low power. A major technology that
we plan to implement is Shape Memory Alloy(SMA). SMA is a new material that can
contract when heated this allows us to use it as an actuator. We can utilize
wires of SMA to control joints and other areas of the hand where space is very
limited. Combining these various technologies, we can make a hand that is very
human-like.

\bibliographystyle{plain}
\bibliography{/home/wenbo/Dropbox/UCB/295/MEngPaper/CapstoneCitations}

\end{document}


